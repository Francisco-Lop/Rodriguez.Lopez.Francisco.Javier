\documentclass[12pt]{article}
%Gummi|065|=)
\usepackage{graphicx}
\usepackage{url}
\title{\textbf{"Caracteristicas de los Conversores AC-DC, DC-AC, DC-DC, AC-AC"}}
\author{Rodriguez Lopez Francisco Javier}
\date{17 de septiembre del 2019}

\begin{document}

\maketitle

\section{Conversores}

Se sabe que un conversor o un convertidor, como su nombre lo da a entender es un dispositivo, el cual tiene la capacidad de hacer un tipo de señales, como las son  analogicas o digitales y darle en viceversa, como lo da su primordial funcion.Proceso en el cual se puede pasa de una energia a otra, de la cual nos ayude en nuestro entorno, de procesos industriales, mas que en si mecanicos, electricos, y electronicos, lo cual nos facilite en situaciones de alto voltaje y pequeña corriente.



\section{CA-CD}
Estos convertidores son los mas utilizados, por la comunidad, puesto que son los que nos dejan conectar nuestros dispositivos caseros, e incluso algunos industriales, en ocasiones, conviertiendo la potencia de energia alterna directa, siendo esto algo de lo cual nos deja poder hacer funcionar nuestros dispositivos, que no necesiten un voltaje tan alto, y una corriente mediana. El convertidor nos puede proporcionar una señal de salida rectificada, esto es muy constante, pero no lo es tan preciso. cambiando de ondas senoidales completas, a solo una onda recta, que en si no es recta. Puesto que se aprecia un ligero manejo de curvas sobre esa onda, esto porque como se dijo no es tan constante el cambio de voltaje, sino que varia para poder tener mejor manejo sobre nuestros componentes.
Algunos de los aspectos a tomar de este cambio de energia es:

-Conectados directamente a un toma-corrientes.

-Manejamiento de un voltaje entre los 5V a 12V.

-La corriente debe de ser por lo menos de 1 amperaje.

-Conversion de una red alterna de la red de distribucion continua llamada Rectificacion.

-Se necesita de estabilizacion, para la buena utilizacion de los componentes.

-Utilizacion de diodos rectificadores, tranbsistores, para el acomodo de el voltaje.

-La corriente de salida, puede tener ambas direcciones y asumirse como constante de magnitud debido a la carga inductiva que tiene en si.

-La rectificacion ayuda, para que la produccion de señales de corriente alterna, cuya frecuencia y magnitud, puedan ser controladas.


\section{CD-CA}
En este caso, para poder convertir la energia de potencia continua o directa a alterna, es necesario usar un inversor, o transformador, lo cual hace que el voltaje de corriente directa, ya sea una medida de 12 voltios, esos mismos 12 voltios salgan pero de manera aterna, dando el mismo valor de entrada y salida. Para que esto sea mas explicito el transformador tiene una cierta capacidad de hacer que ese voltaje salga completo y no haya fallas de por medio y es el de poner dos bobinas, una conectada a su componente de energia alterna y otra bobina que vaya a la corriente continua, una vez hace esto, las bobinas crearan un campo magnetico en cual funcionara como disipador de energia, el cual funciona con transistores la mayoria de casos, que al exitarse con el campo magentico creado por los pulsos de las bobinas, estos hagan llegar el voltaje que se requiera en el lado alterno,  todo esto llegaria al componente de la misma forma que salio en el comienzo.

 -Se necesita de rectificadores en la parte transformada.
 
 -Acompañado de un bloque de potencia en el sistema, para la generacion de una onda alterna, a partir de la suministracion de una fuente de DC.
 
 -Conectados a dispositivos de rapida conmutacion, para la generacion de la onda alterna.
 
 -Control para la buena generacion de señales, que hagan la activacion y desactivacion de los transistores de potencia.
 
 -Primordial accion, la de un transformador el cual ayude la regeneracion de la señal de carga, mandando una onda senoidal, con calibraciones.
 
 -Requerimiento de un mosfet, el cual ayuda al llamado de la fuente de corriente continua, con la transformacion en este caso alterno. 
 
 
 \section{CD-CD}
 Esta conversion aunque a simple lectura, se entiende como algo ilogico ya que la corriente continua siempre va a ser corriente continua, pero en este caso la corriente continua se puede cambiar para adaptarla a otra coorriente continua. Cuando se dice eso, se refiere a interferir entre el voltaje en un parte en los casos mas comunes de la fuente de DC, que se esta manejando y la otra parte, de la cual se quiere adaptar para un voltaje, o en otro cambio del cual seria un mayor amperaje. ESto se logra gracias a un convertidor, en este caso de mayor preferencia de voltaje ya que es el mas comun, que se encuentra, y con menor complejidad de uso, esto cambiando el DC-DC, de una tension a otralogrado tambien en mayor parte por reguladores de conmutacion dando un asalida de voltaje regulada por dichos componentes, y con una corriente limitada.
 
 -Permite generar las tensiones donde estas se requieran, donde se reducen la cantidad de lineas de potencia necesarias.
 
 -Mayor manejo de potencia, mejor manejo de la seguridad del voltaje y control sobre las tensiones.
 
 -Existen tres esquemas de conversiones: 
 
 1-Convertidor reductor (Buck): Su sallida de tension es menor o igual a la tension de entrada.
 
 2-Convertidor elevador (Boost): Este es la inversa del Buck, ya que la tension de salida es igual o superior a la tension de entrada
 3-Convertidor reductor-elevador (Buck-Boost): Este recrea las dos acciones aneriores, en una sola misma pero con la diferencia de que las polaridades seran invertidas, atravez de un ciclo de trabajo.
 
 -Utilizacion de Choppers, aparato conertidor de DC-DC, sirviendo como reductor o elevacion de tension.
 
 -Objetivizando sistemas mas eficientes de menor volumen y menor peso, para la utlizacion en lineas telefonicas, o en sistemas de computo.
 
 
 \section{CA-CA}
 Esta conversion, es similar a la de CD-CD, cumple funciones en constante similares a ellas, pero en este caso con sus propios rangos que lo consta de ser una fuente de corriente alterna, como lo seria la frecuecnia, amplitud, y fase, siendo cada una de ellas algo que se puede ver en algunas partes del mundo, en donde se maneja una frecuencia diferente, y es de ahi su derivado, aunque su aplicaion es en si, enorme y nos deja abarcar con dispositivos mas complejos y de mayor manejo. Empleando esto de manera eficiente, para variar la tension eficaz, a travez de una carga constante o frecuencia, se utiliza controladores o reguladores de voltaje, para la conversion o cambio. Como se ha visto ya, este cambio hace que toda la fuente cambia, pero en si este cambio de da de mejor forma con un circuito monofasico, el cual teniendo todos y cada uno de los componentes, se puede variar la fase, la amplitud, y la freciencia de la fuente a cambiar, esto gracias a rectificadores, y una fuente seleccionada.
 
 -Cotrol de voltaje logrado mediante la fase en virtud de la conmutacion fisica.
 
 -Utilizando rectificadores de silicio o en su caso triac, en otra persepctiva llamados auto-conmutados como tiristores puerta a apagar.
 
 -Utilizacion de ciclo-convertidores, que permiten una conversion directa, tanto de amplitud, como de frecuencia, sin paso por un CD.
 
 -Controlador monofasico de correinte alterna, esto para proporcionar una conexion bidireccional de onda completa, y que el cambio sea homogeneo.
 
 -Se pueden utilizar entre dos tiristores, ya sea en comunes, en catodos, o un solo tiristor.
 
 -Entre mas alta la amplitud y la frecuencia, mas se utiliza una onda senoidal, para poder apreciar el aumento que se le quiere dar, a la fuente CA. 
 
 
 \newpage{}
 \section{Referencias Bibliograficas}
 
 \url{https://es.wikihow.com/convertir-AC-a-DC}\\
 \url{http://catarina.udlap.mx/u_dl_a/tales/documentos/lem/moyaho_l_i/capitulo1.pdf}\\
 \url{http://www.ptolomeo.unam.mx:8080/jspui/nitstream/132.248.52.100/787/8/A8.pdf}\\
 \url{http://catarina.udlap.mx/u_dl_a/tales/documentos/lem/martinez_v_da/capitulo2.pdf}\\
 \url{http://ciep.ing.uaslp.mx/njjccontrol/images/pdf/tema_8.pdf}\\
 \url{http://ocw.uc3m.es/tecnologia-electronica/electronica-de-potencia/material-de-clase-1/MC-F-004.pdf}\\
 \url{http://www.potencia.uma.es/index.php?option=com_content&view=article&id=83%3Acapitulo-8&catid=35%3Ae-book&Itemid=80&lang=es}
 
\end{document}
